\section*{СОДЕРЖАНИЕ}

main.c, main.h – в данном модуле создается массив, вызывается меню

menu.c, menu.h – модуль, в котором зациклено меню

open\_file.c, open\_file.h – модуль, который открывает файл

add\_element\_end.c, add\_element.h – модуль, который добавляет элемент в конец массива 

view\_all\_element.c, view\_all\_element.h – модуль, который просматривает массив из памяти, выводя в таблицу

save\_to\_tsv\_file.c, save\_to\_tsv\_file.h – модуль, который сохраняет массив, в текстовый файл в формате TSV (через табуляцию). Этот же файл можно открыть в Libre Office Calc, MS Excel…

correct\_field.c, correct\_field.h – модуль, который просит элемент, и через зацикленное меню просит какое поле изменить

delete\_element.c, delete\_element.h – модуль, который удаляет элемент, который указал пользователь

clear\_console.c, clear\_console.h – модуль, для очистки консоли

getch.c, getch.h – модуль, который позволяет при вызове получать символ нажатый с клавиатуры

pause\_console.c, pause\_console.h – модуль, который выводит сообщение, чтобы нажать клавишу для продолжения и ждет нажатия

submenu.c, submenu.h – модуль, который вызывает зацикленное под меню

delete\_by\_condition.c, delete\_by\_condition.h – модуль, который удаляет элементы по условию

sort\_elements.c, sort\_elements.h – модуль, который сортирует массив

add\_element\_before.c, add\_element\_before.h – модуль который добавляет элемент в массив перед указанным элементом

\newpage

Диск: