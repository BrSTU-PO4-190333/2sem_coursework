\newpage

\section{РАЗРАБОТКА АЛГОРИТМОВ}

\subsection{Алгоритмы общего задания}

\textbf{Алгоритм ввода информации из текстового файла в массив}

\underline{Исходные данные}: текстовый файл, массив

\underline{Алгоритм}:

\begin{enumerate}
    \item Просим файл
    \item Открытие файла в режиме чтения.
    \item Если файл не открыт:
    \begin{enumerate}
        \item Сообщили
    \end{enumerate}
    \item Иначе
    \begin{enumerate}
        \item Цикл, пока не достигнут конец файла:
        \begin{enumerate}
            \item Считываем данные записываем в массив
        \end{enumerate}
        \item Закрытие файла
    \end{enumerate}
\end{enumerate}

\underline{Выходные данные}: сформированный массив

\hspace{0pt}\\



\textbf{Алгоритм добавление нового элементов в конец массива}

\underline{Исходные данные}: массив

\underline{Алгоритм}:

\begin{enumerate}
    \item Увеличение массива
    \item Запись элемента
\end{enumerate}

\underline{Выходные данные}: обновленный массив

\hspace{0pt}\\



\textbf{Алгоритм просмотр элементов}

\underline{Исходные данные}: массив

\underline{Алгоритм}:

\begin{enumerate}
    \item Вывод шапки таблицы
    \item Цикличный вывод элементов до конца строки
\end{enumerate}

\underline{Выходные данные}: нет

\hspace{0pt}\\



\textbf{Алгоритм вывода информации в текстовый файл}

\underline{Исходные данные}: массив

\underline{Алгоритм}:

\begin{enumerate}
    \item Открытие файла в режиме записи
    \item Если файл не открылся
    \begin{enumerate}
        \item сообщить
    \end{enumerate}
    \item Иначе
    \begin{enumerate}
        \item Циклично вывести элементы в файл
        \item Закрыли файл
    \end{enumerate}
\end{enumerate}

\underline{Выходные данные}: текстовый файл

\hspace{0pt}\\



\textbf{Алгоритм корректировки полей}

\underline{Исходные данные}: массив

\underline{Алгоритм}:

\begin{enumerate}
    \item Просим элемент изменения
    \item Изменяем поля в этом элементе
\end{enumerate}

\underline{Выходные данные}: массив измененный

\hspace{0pt}\\



\textbf{Алгоритм удаления выбранного элемента}

\underline{Исходные данные}: массив

\underline{Алгоритм}:

\begin{enumerate}
    \item Просим элемент изменения
    \item Изменяем поля в этом элементе
\end{enumerate}

\underline{Выходные данные}: массив измененный



\subsection{Алгоритмы задания по варианту}

\textbf{Алгоритм удаления элементов по условию}

\underline{Исходные данные}: массив

\underline{Алгоритм}:

\begin{enumerate}
    \item Запуск зацикленного меню, которые вызывает функции
    \item Как получили поле редактирование, меньше (больше) чего удалить
    \item Цикл прохода
    \begin{enumerate}
        \item Если поле меньше числа
        \begin{enumerate}
            \item цикл со смешением улементов
            \item уменьшение массива
        \end{enumerate}
    \end{enumerate}
\end{enumerate}

\underline{Выходные данные}: массив измененный

\hspace{0pt}\\



\textbf{Алгоритм сортировки}

\underline{Исходные данные}: массив

\underline{Алгоритм}:

\begin{enumerate}
    \item Запуск зацикленного меню, которое вызывает функции
    \item Как получили команду, то сортируем по полю через буферный обмен
\end{enumerate}

\underline{Выходные данные}: массив измененный

\hspace{0pt}\\



\textbf{Алгоритм вставки элемента перед выбраным}

\underline{Исходные данные}: массив

\underline{Алгоритм}:

\begin{enumerate}
    \item Просим перед каким элементом вставить
    \item Увеличиваем массив
    \item В цикле смещаем элементы
    \item Редактируем поле, куда хотели вставить
\end{enumerate}

\underline{Выходные данные}: массив измененный

\newpage